\documentclass[a4paper]{scrartcl}

% makes 68 colors available
\usepackage[usenames,dvipsnames]{xcolor}

\usepackage[latin1]{inputenc}

% sets english to the first language and german to the second laguage
\usepackage[ngerman,english]{babel}

% verbatim package to use block comments
\usepackage{verbatim}

% graphicx package to include graphics
\usepackage{graphicx}
\graphicspath{ {./graphics/} }

% Force LaTeX to use in the whole document the
% standard Computer modern font (if no other font is specified).
% This is done to prevent loaded fonts to overwrite the standard font.
\renewcommand*\rmdefault{cmr}
\renewcommand*\sfdefault{cmss}
\renewcommand*\ttdefault{cmtt}

% include more packages here
\usepackage{amsmath}


% placeins package to use the \FloatBarrier command
\usepackage[section]{placeins}


% listings package for including source code
\usepackage{listings}

% insert style definitions here
% a style for Matlab
\usepackage{color} %red, green, blue, yellow, cyan, magenta, black, white
\definecolor{matlabgreen}{RGB}{28,172,0} % color values Red, Green, Blue
\definecolor{matlablila}{RGB}{170,55,241}

\lstdefinestyle{MyMatStyle}{
language = Matlab,
%basicstyle=\color{red},
basicstyle=\fontfamily{cmtt}\selectfont,
breaklines=true,%
morekeywords={matlab2tikz},
keywordstyle=\color{blue},%
morekeywords=[2]{1}, keywordstyle=[2]{\color{black}},
identifierstyle=\color{black},%
stringstyle=\color{matlablila},
commentstyle=\color{matlabgreen},%
showstringspaces=false,
numbers=none,%
numberstyle={\tiny \color{black}},% size of the numbers
numbersep=9pt, % this defines how far the numbers are from the text
%some words to emphasise
emph=[1]{for,end,break},emphstyle=[1]\color{red},
%emph=[2]{word1,word2}, emphstyle=[2]{style},
columns = fullflexible
}


% my own command for making comments in a LaTex project
\newcommand{\tp}[1]{\textcolor{green}{TP: "#1"}}

% define new commands here
\newcommand{\matinl}[1]{
\lstinline[style=MyMatStyle,keywordstyle=\color{black},
stringstyle=keywordstyle=\color{black},commentstyle=\color{black},
numberstyle=\color{black}]{#1}
}


\begin{document}

% sets the indent of the first line to 0 cm
\setlength{\parindent}{0pt}
\title{HLCV Exercise 2}


\maketitle
\section*{Question 3d}
For the graffity image, both dxdy-hist and rg-hist give good results. Most of the found matchings are correct. 

We see, that for the correspondencies of the New-York image, MagLap-hist performs much better then dxdy-hist. This is because the Laplacian and the Gradient Magnitude of a function are invariant under rotations, while the gradient is not. Although, not every found match is right.

Additionally, I found that both Hessian and Harris detectors work equally well.
\section*{Question 4a}
Homography estimation has the general purpose of finding a $3 \times 
3$-projection matrix for a projective mapping between two images in homogenous 
coordinates. Since the scale factor of our matrix is set to 1 we have to 
estimate 8 unknown parameters:
\begin{equation*}
\begin{pmatrix}
h_{11} & h_{12} & h_{13}\\
h_{21} & h_{22} & h_{23}\\
h_{31} & h_{32} & 1\\
\end{pmatrix}
\end{equation*}
The naive idea is to set a system of equations using point matches between 
$(x_i,y_i), (x_i',y_i')$ which would look like this
\begin{equation*}
\begin{pmatrix}
x_i & y_i & 0 & 0 & 0 & 0 & 1 & 0 & 0\\
0 & 0 & x_i & y_i & 0 & 0 & 0 & 1 & 0\\
0 & 0 & 0 & 0 & x_i & y_i & 0 & 0 & 1\\
\end{pmatrix}
\begin{pmatrix}
h_{11}\\
h_{12}\\
h_{21}\\
h_{22}\\
h_{31}\\
h_{32}\\
h_{13}\\
h_{23}\\
1
\end{pmatrix}
=
\begin{pmatrix}
x_i'\\[3pt]
y_i'\\[3pt]
1
\end{pmatrix}
\end{equation*}
$\Rightarrow$ we get 3 equations for our 8 unknowns from one point match.\\
If we extend the system with a sufficient number of points matches it becomes 
overspecified so we cannot solve it directly. So the idea would be using a 
least-squares fit.\\
The problem now is that wrong point matches (outliers) can totally spoil the 
estimation of our parameters since the least-squares fit tries 
to approximate all our point matches and punishes distance quadratically.\\
To overcome this drawback we use the RANSAC algorithm: The key idea is to 
select $n$ random points and rate whether we found a set of ``inliers''. In 
order to do this we compute the homography matrix approximating our selected 
points using a least-squares fit then we have a tolerance region around our 
fit. All the points inside that tolerance are counted as inliers. We repeat the 
previous step $k$ times and select the best point set with respect to the 
number of inliers it contains.\\
Now to provide an estimate how many times we have to sample (paramter $k$) to 
find a set of inliers given a failure probability $p$ we can view the process 
as an Bernoulli process:\\
$w$ is the probability to get an inlier\\
$n$ is the sample size = length of the Bernoulli process\\
$k$ is the number of samples = number of Bernoulli processes\\
Given this it is easy to see that the probability that all samples fail is
\begin{equation*}
p = (1 - w^n)^k
\end{equation*}
We use this formula to adapt $k$.
\bigskip


Now we can take a look how this is happening in the code:
The \matinl{get_ransac_hom} function takes two lists of point coordinates 
\matinl{x1,y1,x2,y2} and two images \matinl{img1,img2} and retuns our 
projection matrix \matinl{H}.\\
In the code segment
\begin{lstlisting}[style=MyMatStyle,float=!h]
sample_size = size(x1,1); % we have sample_size samples
pFail = 0.001; % we want to be 99.9 percent sure
pInlier = 4 / sample_size; % just a very conservative guess for the amount of inlier (only 4);
\end{lstlisting}
\FloatBarrier
we set our important parameters w to $4/\text{sample size}$ and p to $0.1\%$.\\
We convert then our Cartesian coordinates to homogenous coordinates before we 
start with our RANSAC iterations.\\
In the present implementation we have a sample 4 random point matches and 
calculate their matrix \matinl{H} using
\begin{lstlisting}[style=MyMatStyle,float=!h]
pos = uint32 (rand(4,1)' .* [sample_size-1, sample_size-1, sample_size-1, sample_size-1] + [1,1,1,1]);
testH = get_hom( x1(pos), y1(pos), x2(pos), y2(pos) );
\end{lstlisting}
\FloatBarrier
We use then the matrix H and count the number of inlier within our given 
tolerance as explained above.\\
A slight modification is that we readjust the number of tests that we perform by
\begin{lstlisting}[style=MyMatStyle,float=!h]
      % adjust maximal iterations to be made
      % you can reestimate the probability for picking a true pair by
      % looking at the percentage of inliers of the current estimation
      pCurrentInlier = inlierCount / sample_size;

      if ( pCurrentInlier > pBestInlier)

        % find the amount of times k we have to draw a sample without outliers
        kTests = log(pFail) / log(1 - pCurrentInlier^4);
        toDoTests = round(kTests);

        fprintf('proportion of inliers: %f, number of RANSAC interations: %d\n', pCurrentInlier, toDoTests);

        % best Homography and amount of inliers
        H = testH;
        pBestInlier = pCurrentInlier;
      end

      if (toDoTests < j ) 
          break; % end here, we are pFail percent sure to have found H
      end
\end{lstlisting}
\FloatBarrier
In the end we collect all our inlier within the tolerance of our best chosen 
point match set and recalculate the \matinl{H} matrix using them:
\begin{lstlisting}[style=MyMatStyle,float=!h]
% probability of picking an inlier from the parameter set
   pInlier = pBestInlier; 
     
   % first find the inliers
   hitId = zeros(sample_size,1);
   for i = 1:sample_size
     p = H * [x1(i), y1(i), 1.0]';
     p = p / p(3);
     if (norm (p(1:2) - [x2(i),y2(i)]') < distInlier)
       hitId(i) = i;
     end
   end
   hitId = find (hitId);

   % re-estimate Homography
   H = get_hom( x1(hitId), y1(hitId), x2(hitId), y2(hitId) );
\end{lstlisting}
\FloatBarrier
The division by the last component \matinl{p = p / p(3);} has to be made because
we want to normalize our homogenous coordinates such that they have all scaling 
factor 1. Only then points can be compared among eachother.

     



\end{document}